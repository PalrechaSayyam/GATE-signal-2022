\iffalse
\let\negmedspace\undefined
\let\negthickspace\undefined
\documentclass[journal,12pt,twocolumn]{IEEEtran}
\usepackage{cite}
\usepackage{amsmath,amssymb,amsfonts,amsthm}
\usepackage{algorithmic}
\usepackage{graphicx}
\usepackage{textcomp}
\usepackage{xcolor}
\usepackage{txfonts}
\usepackage{listings}
\usepackage{enumitem}
\usepackage{mathtools}
\usepackage{gensymb}
\usepackage{comment}
\usepackage[breaklinks=true]{hyperref}
\usepackage{tkz-euclide} 
\usepackage{listings}
\usepackage{gvv}  
\usepackage{tikz}
\usepackage{circuitikz} 
\usepackage{caption}
\def\inputGnumericTable{}              
\usepackage[latin1]{inputenc}          
\usepackage{color}                    
\usepackage{array}                     
\usepackage{longtable}                 
\usepackage{calc}                     \usepackage{multirow}                  
\usepackage{hhline}                    
\usepackage{ifthen}                    
\usepackage{lscape}
\usepackage{amsmath}
\newtheorem{theorem}{Theorem}[section]
\newtheorem{problem}{Problem}
\newtheorem{proposition}{Proposition}[section]
\newtheorem{lemma}{Lemma}[section]
\newtheorem{corollary}[theorem]{Corollary}
\newtheorem{example}{Example}[section]
\newtheorem{definition}[problem]{Definition}
\newcommand{\BEQA}{\begin{eqnarray}}
\newcommand{\EEQA}{\end{eqnarray}}
\newcommand{\define}{\stackrel{\triangle}{=}}
\theoremstyle{remark}
\newtheorem{rem}{Remark}

%\bibliographystyle{ieeetr}
\begin{document}
%

\bibliographystyle{IEEEtran}




\title{
%	\logo{
G.A.T.E.

\large{EE1205 : Signals and Systems}

Indian Institute of Technology Hyderabad
%	}
}
\author{Chirag Garg

(EE23BTECH11206)
}	





\maketitle

\newpage



\bigskip

\renewcommand{\thefigure}{\theenumi}
\renewcommand{\thetable}{\theenumi}


\section{Question E.E.(32)}
\vspace{0.5cm}



\textbf{Question:} 
Let $f(t)$ be an even function, i.e.$f(-t) = f(t)$ for all t.Let the Fourier transform of $f(t)$ be defined as $F(\omega) = \int_{-\infty}^{\infty} f(t) e^{-j \omega t} \, dt $ . Suppose $\dfrac{dF(\omega)}{d \omega} = -\omega F(\omega)$ for all $\omega$ , and $F(0) = 1$ . Then


\begin{enumerate}[label = (\Alph*)]
\item $f(0) < 1 $\\
\item  $f(0) > 1 $\\
\item  $f(0) = 1 $\\
\item   $f(0) = 0 $\\
\end{enumerate} \hfill{(GATE EE 2021)}\\
%\section{Solution} 
\textbf{Solution: }
\fi
Given, \begin{align}
\dfrac{dF(\omega)}{d \omega} &= -\omega F(\omega) \\
\dfrac{dF(\omega)}{d \omega} + \omega F(\omega) &= 0 \\
ln|F(\omega)| &= -\dfrac{\omega^{2}}{2} + c \\
F(\omega) &= Ke^{-\frac{\omega^2}{2}}
\end{align}
Put $\omega = 0$ , \begin{align}
F(0) &= K \\
K&=1
\end{align}
\begin{align}
\therefore F(\omega) &= e^{-\frac{\omega^2}{2}}
\end{align}

\begin{center}
 $f(t) \longleftrightarrow F(\omega)$
\end{center}
 \begin{align}
e^{-at^{2}}  \longleftrightarrow  \sqrt{\dfrac{\pi}{a}}
e^{-\frac{\omega^2}{4a}} \; ; \; a > 0 \\
At \; a = \dfrac{1}{2} , \; e^{-\frac{t^{2}}{2}}  \longleftrightarrow  \sqrt{2\pi} e^{-\frac{\omega^2}{2}}\\
\dfrac{1}{\sqrt{2\pi}}e^{-\frac{t^{2}}{2}}  \longleftrightarrow  e^{-\frac{\omega^2}{2}} = F(\omega) \\
\text{Thus , } f(t) = \dfrac{1}{\sqrt{2\pi}}e^{-\frac{t^{2}}{2}} 
 \end{align}
 At $t = 0$  \begin{align}
 f(0) = \dfrac{1}{\sqrt{2\pi}} < 1
 \end{align}
 Hence , option (a) is correct.
