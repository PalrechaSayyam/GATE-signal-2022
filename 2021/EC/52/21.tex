\iffalse
\let\negmedspace\undefined
\let\negthickspace\undefined
\documentclass[a4,12pt,onecolumn]{IEEEtran}
\usepackage{amsmath,amssymb,amsfonts,amsthm}
\usepackage{algorithmic}
\usepackage{graphicx}
\usepackage{textcomp}
\usepackage{xcolor}
\usepackage{txfonts}
\usepackage{listings}
\usepackage{enumitem}
\usepackage{mathtools}
\usepackage{gensymb}
\usepackage[breaklinks=true]{hyperref}
\usepackage{tkz-euclide}
\usepackage{listings}
\usepackage{circuitikz}

\begin{document}

\title{ signals and systems
\vspace{1cm}

Gate2021-ec-Q52}
\author{EE23BTECH11014- Devarakonda Guna vaishnavi}
\maketitle
\textbf{Question:}

A message signal having peak-to-peak value of $2 \, \text{V}$, root mean square value of $0.1 \, \text{V}$, and bandwidth of $5 \, \text{kHz}$ is sampled and fed to a pulse code modulation (PCM) system that uses a uniform quantizer. The PCM output is transmitted over a channel that can support a maximum transmission rate of $50 \, \text{kbps}$. Assuming that the quantization error is uniformly distributed, calculate the maximum signal-to-quantization noise ratio (rounded off to two decimal places).\\
\solution:
\fi
\begin{table}[h!]
    \centering
    

\begin{center}
\begin{tabular}{|c|p{6cm}|c|c|}
\hline
\textbf{Term} & \textbf{Description} & \textbf{Formula} & \textbf{Value} \\
\hline
Peak-to-Peak & Peak-to-peak value of the message signal (in volts). & $\sqrt{2} \times \text{RMS}$ & $1.414 \, \text{V}$ \\
\hline
RMS & Root mean square value of the message signal (in volts). & $\frac{2}{2\sqrt{2}}$ & $0.707 \, \text{V}$ \\
\hline
Bandwidth & Bandwidth of the message signal (in kilohertz). & - & - \\
\hline
N & Number of bits per sample used in quantization. & - & $5$ \\
\hline
Quantization range & Range of values in which the quantized signal falls (in volts). & $2 \times \text{Peak-to-Peak}$ & $2.828 \, \text{V}$ \\
\hline
Step size & The size of each quantization interval (in volts). & $\frac{\text{Quantization range}}{2^N}$ & - \\
\hline
Sampling rate & Rate at which the message signal is sampled (in kilohertz). & - & $10 \, \text{kHz}$ \\
\hline
Maximum transmission rate & Maximum transmission rate supported by the channel (in kilobits per second). & - & $50 \, \text{kbps}$ \\
\hline
\end{tabular}
\end{center}


    \caption{Input Parameters}
    \label{table:parameters}
\end{table}\\
\textbf{Calculate SQNR(Signal-to-quantization noise ratio):}
\begin{align} 
\text{SQNR} &= 6.02 \times N + 1.76 
\label{eq:sqnr}\\
\implies \text{SQNR} &= 31.86
\end{align}
\end{document}
