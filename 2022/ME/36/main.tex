\iffalse
\let\negmedspace\undefined
\let\negthickspace\undefined
\documentclass[journal,12pt,onecolumn]{IEEEtran}
\usepackage{cite}
\usepackage{amsmath,amssymb,amsfonts,amsthm}
\usepackage{algorithmic}
\usepackage{graphicx}
\usepackage{textcomp}
\usepackage{xcolor}
\usepackage{txfonts}
\usepackage{listings}
\usepackage{enumitem}
\usepackage{mathtools}
\usepackage{gensymb}
\usepackage{comment}
\usepackage[breaklinks=true]{hyperref}
\usepackage{tkz-euclide} 
\usepackage{listings}
\usepackage{gvv}                                        
\def\inputGnumericTable{}                                 
\usepackage[latin1]{inputenc}                                
\usepackage{color}                                            
\usepackage{array}                                            
\usepackage{longtable}                                       
\usepackage{calc}                                             
\usepackage{multirow}                                         
\usepackage{hhline}                                           
\usepackage{ifthen}                                           
\usepackage{lscape}

\newtheorem{theorem}{Theorem}[section]
\newtheorem{problem}{Problem}
\newtheorem{proposition}{Proposition}[section]
\newtheorem{lemma}{Lemma}[section]
\newtheorem{corollary}[theorem]{Corollary}
\newtheorem{example}{Example}[section]
\newtheorem{definition}[problem]{Definition}
\newcommand{\BEQA}{\begin{eqnarray}}
 \newcommand{\EEQA}{\end{eqnarray}}
\newcommand{\define}{\stackrel{\triangle}{=}}
\theoremstyle{remark}
\newtheorem{rem}{Remark}
\begin{document}
 \bibliographystyle{IEEEtran}
 \vspace{3cm}
 \title{\textbf{ME 36}}
 \author{EE23BTECH11048-Ponugumati Venkata Chanakya$^{*}$% <-this % stops a space
 }
 \maketitle

 \bigskip
 \renewcommand{\thefigure}{\theenumi}
 \renewcommand{\thetable}{\theenumi}
 \textbf{QUESTION:}
 The value of Integral \\
  \begin{align*}
        \oint \brak{\frac{6z}{2z^4-3z^3+7z^2-3z+5}}dz
 \end{align*}
 evaluated over a counter-clockwise circular contour in the complex plane enclosing only the pole $z=\jmath $, where $\jmath$ is the imaginary unit,is
 \begin{enumerate}
     \item $(-1+\jmath)\pi$
     \item $(1+\jmath)\pi$
     \item $2(1-\jmath)\pi$
     \item $(2+\jmath)\pi$
 \end{enumerate}
 \hfill{(GATE 2022 ME)}\\
 \solution\\
\fi
 Given $z=\jmath$ is only enclosing pole 
 \begin{align}
 \oint \brak{\frac{6z}{2z^4-3z^3+7z^2-3z+5}}dz
 &=\oint \brak{\frac{\frac{6z}{2z^3+(2\jmath-3)z^2+(5-3\jmath)z+5\jmath}}{z-\jmath}}dz\\
 &=2\pi \jmath\brak{\frac{6z}{2z^3+(2\jmath-3)z^2+(5-3\jmath)z+5\jmath}} \text{ At }z=\jmath
    \text{ (Cauchy's integral formula)}\\
    &=2\pi \jmath\brak{\frac{6\jmath}{2\jmath^3+(2\jmath-3)\jmath^2+(5-3\jmath)\jmath+5\jmath}}\\
    &=2\pi \jmath\brak{\frac{\jmath}{\jmath+1}}\\
    &=-2\pi \frac{\jmath-1}{\jmath^2-1}\\
    &=(-1+\jmath)\pi
    \end{align}
 %\end{document}

