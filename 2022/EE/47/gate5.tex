\iffalse
\let\negmedspace\undefined
\let\negthickspace\undefined
\documentclass[journal,12pt,twocolumn]{IEEEtran}
\usepackage{cite}
\usepackage{amsmath,amssymb,amsfonts,amsthm}
\usepackage{algorithmic}
\usepackage{graphicx}
\usepackage{textcomp}
\usepackage{xcolor}
\usepackage{pgfplots}
\usepackage{txfonts}
\usepackage{listings}
\usepackage{enumitem}
\usepackage{mathtools}
\usepackage{gensymb}
\usepackage{comment}
\usepackage[breaklinks=true]{hyperref}
\usepackage{tkz-euclide} 
\usepackage{listings}
\usepackage{gvv}                                        
\def\inputGnumericTable{}                                 
\usepackage[latin1]{inputenc}                                
\usepackage{color}                                            
\usepackage{array}                                            
\usepackage{longtable}                                       
\usepackage{calc}                                             
\usepackage{multirow}                                         
\usepackage{hhline}                                           
\usepackage{ifthen}                                           
\usepackage{lscape}

\newtheorem{theorem}{Theorem}[section]
\newtheorem{problem}{Problem}
\newtheorem{proposition}{Proposition}[section]
\newtheorem{lemma}{Lemma}[section]
\newtheorem{corollary}[theorem]{Corollary}
\newtheorem{example}{Example}[section]
\newtheorem{definition}[problem]{Definition}
\newcommand{\BEQA}{\begin{eqnarray}}
\newcommand{\EEQA}{\end{eqnarray}}
\newcommand{\define}{\stackrel{\triangle}{=}}
\theoremstyle{remark}
\newtheorem{rem}{Remark}
\begin{document}
\parindent 0px
\bibliographystyle{IEEEtran}
\title{GATE: EE - 47.2022}
\author{EE22BTECH11219 - Rada Sai Sujan$^{}$% <-this % stops a space
}
\maketitle
\newpage
\bigskip
\section*{Question}
Let an input $x\brak{t}=2\sin (10\pi t )+5\cos (15\pi t)+7\sin(42\pi t)+4\cos (45\pi t)$ is passed through an LTI system having an impulse response $$h\brak{t}=2\brak{\frac{\sin\brak{10\pi t}}{\pi t}}\cos \brak{40\pi t}$$ The output of the system is \\
\begin{enumerate}[label=(\alph*)]
    \item $2\sin\brak{10\pi t}+5\cos\brak{15\pi t}$
    \item $2\sin\brak{10\pi t}+4\cos\brak{45\pi t}$
    \item $7\sin\brak{42\pi t}+4\cos\brak{45\pi t}$
    \item $5\sin\brak{15\pi t}+7\cos\brak{42\pi t}$
\end{enumerate}
\solution
\fi

\begin{table}[ht]
    \centering
    \begin{tabular}{|p{4cm}|p{2.8cm}|}
    \hline
    Frequency components of input & Value   \\ \hline 
    $$f_1$$ & $$\frac{10\pi}{2\pi}=5Hz$$ \\ \hline
    $$f_2$$ & $$\frac{15\pi}{2\pi}=7.5Hz$$  \\ \hline
    $$f_3$$ & $$\frac{42\pi}{2\pi}=21Hz$$  \\ \hline
    $$f_4$$ & $$\frac{45\pi}{2\pi}=22.5Hz$$    \\ \hline
\end{tabular}

    \caption{Frequency components}
    \label{tab:gate22ee47Q.1}
\end{table}
Given,
\begin{align}
    h\brak{t}&=2\brak{\frac{\sin\brak{10\pi t}}{\pi t}}\cos \brak{40\pi t}  \\
    &=\frac{\sin 50\pi t}{\pi t}-\frac{\sin 30\pi t}{\pi t} \\
    &=h_1\brak{t}-h_2\brak{t}
\end{align}
where,
\begin{align}
    h_1\brak{t} &= \frac{\sin 50\pi t}{\pi t}   \\
    h_2\brak{t} &= \frac{\sin 30\pi t}{\pi t}
\end{align}
Taking Fourier transform of $h\brak{t}$
\begin{align}
    h\brak{t} &\overset{\mathcal{F}}{\longleftrightarrow} H_1\brak{f}-H_2\brak{f}
\end{align}
where,
\begin{align}
    h_1\brak{t} &\overset{\mathcal{F}}{\longleftrightarrow} H_1\brak{f}  \\
    h_2\brak{t} &\overset{\mathcal{F}}{\longleftrightarrow} H_2\brak{f}  
\end{align}
Plotting $H_1\brak{f}$ and $H_2\brak{f}$ we get,    \\
\begin{align}
\begin{tikzpicture}
\begin{axis}[
    axis lines=middle,
    xmin=-40,
    xmax=40,
    ymin=-0.5,
    ymax=1.5,
    xlabel={$f$},
    ylabel={$H_1(f)$},
    xtick={-25,25},
    ytick={0,1},
    ]
    \addplot [blue, thick] coordinates {(-35,0)(-25,0) (-25,1) (25,1) (25,0)(35,0)};
\end{axis}
\end{tikzpicture}   \label{kk:gateee47Q.1} \\
\begin{tikzpicture}
\begin{axis}[
    axis lines=middle,
    xmin=-30,
    xmax=30,
    ymin=-0.5,
    ymax=1.5,
    xlabel={$f$},
    ylabel={$H_2(f)$},
    xtick={-15,15},
    ytick={0,1},
    ]
    \addplot [blue, thick] coordinates {(-25,0)(-15,0) (-15,1) (15,1) (15,0)(25,0)};
\end{axis}
\end{tikzpicture}   \label{kk:gateee47Q.2}
\end{align}
Plotting $H\brak{f}$ from \figref{kk:gateee47Q.1} and \figref{kk:gateee47Q.2}
\begin{tikzpicture}
\begin{axis}[
    axis lines=middle,
    xmin=-40,
    xmax=40,
    ymin=-0.5,
    ymax=3,
    xlabel={$f$},
    ylabel={$H(f)$},
    xtick={-25,-15,15,25},
    ytick={0,1},
    ]
    \addplot [blue, thick] coordinates {(-35,0)(-25,0)(-25,1)(-15,1)(-15,0)(15,0)(15,1)(25,1)(25,0)(35,0)};
\end{axis}
\end{tikzpicture}   \\
Therefore, the given system is a Bandpass filter with passband: \\
\begin{align}
    15\leq\abs{f}\leq 25    \label{equation:gate22ee47Q.1}
\end{align}

Veryfying \tabref{tab:gate22ee47Q.1} with \eqref{equation:gate22ee47Q.1}, only $f_3$ and $f_4$ will be passed through the system. \\
\begin{align}
    \therefore y\brak{t}=7\sin\brak{42\pi t}+4\cos\brak{45\pi t}
\end{align}
($\because\abs{H\brak{f}}=1$, the amplitude of frequency components will be unchanged.)
