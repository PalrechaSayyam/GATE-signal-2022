\begin{enumerate}[label=\thechapter.\arabic*,ref=\thechapter.\theenumi]

\item The damping ratio and undamped natural frequency of a closed loop system as
shown in the figure, are denoted as $\zeta$ and $\omega_n$, respectively. The values of $\zeta$ and $\omega_n$
are 
\begin{figure}[!ht]
\centering
\begin{center}
\includegraphics[width=\columnwidth]{2022/EE/39/figs/question.jpg}
\end{center}
%\caption{Diagram for GATE ME Question 30}
\end{figure}
\begin{enumerate}
    \item $\zeta = 0.5$ and $\omega_n = 10$ rad/s
    \item $\zeta = 0.1$ and $\omega_n = 10$ rad/s
    \item $\zeta = 0.707$ and $\omega_n = 10$ rad/s
    \item $\zeta = 0.707$ and $\omega_n = 100$ rad/s
\end{enumerate}
\hfill(GATE EE 2022)
\solution
\input{2022/EE/39/ee39.tex}
\newpage
\item In the block diagram shown in the figure, the transfer function $G=\frac{K}{\tau s+1}$ with $K>0$ and $\tau>0$. The maximum value of $K$ below which the system remains stable is \rule{1cm}{0.15mm}(rounded off to two decimal places) \hfill (GATE CH 2022) 
\begin{figure}[htbp] 
\includegraphics[width=\columnwidth]{2022/CH/58/figs/question.jpg} 
\end{figure}\\ 
\solution 
\input{2022/CH/58/ch_58.tex} 
\newpage
\item The output of the system y\brak{t} is related to its input x\brak{t} according to the relation $y\brak{t}=x\brak{t}sin\brak{2\pi t}$.This system is 
\renewcommand{\labelenumi}{\alph{enumi})}
\begin{enumerate}
\item Linear and time-variant
\item Non-Linear and time-invariant
\item Linear and time-invariant
\item Non-linear and time-variant
\end{enumerate}
\hfill{(GATE 2022 IN 14)}
\solution
\input{2022/IN/14/assign5.tex}
\newpage
\end{enumerate}
