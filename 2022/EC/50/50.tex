\documentclass[journal,12pt,onecolumn]{IEEEtran}
\usepackage{cite}
\usepackage{amsmath,amssymb,amsfonts,amsthm}
\usepackage{algorithmic}
\usepackage{graphicx}
\usepackage{textcomp}
\usepackage{xcolor}
\usepackage{txfonts}
\usepackage{listings}
\usepackage{enumitem}
\usepackage{mathtools}
\usepackage{gensymb}
\usepackage{comment}
\usepackage[breaklinks=true]{hyperref}
\usepackage{tkz-euclide}
\usepackage{listings}
\usepackage{gvv}
\def\inputGnumericTable{}
\usepackage[latin1]{inputenc}
\usepackage{color}
\usepackage{array}
\usepackage{longtable}
\usepackage{calc}
\usepackage{multirow}
\usepackage{hhline}
\usepackage{ifthen}
\usepackage{lscape}

\newtheorem{theorem}{Theorem}[section]
\newtheorem{problem}{Problem}
\newtheorem{proposition}{Proposition}[section]
\newtheorem{lemma}{Lemma}[section]
\newtheorem{corollary}[theorem]{Corollary}
\newtheorem{example}{Example}[section]
\newtheorem{definition}[problem]{Definition}
\newcommand{\BEQA}{\begin{eqnarray}}
    \newcommand{\EEQA}{\end{eqnarray}}
\newcommand{\define}{\stackrel{\triangle}{=}}
\theoremstyle{remark}
\newtheorem{rem}{Remark}

\begin{document}
    
    \bibliographystyle{IEEEtran}
    \vspace{3cm}
    
    \title{Gate 2022 EC Q55}
    \author{EE23BTECH11212 - Manugunta Meghana Sai$^{*}$% <-this % stops a space
    }
    \maketitle
    \bigskip
    
    \renewcommand{\thefigure}{\theenumi}
    \renewcommand{\thetable}{\theenumi}
    
    \vspace{3cm}
    \textbf{Gate 2022 EC Q50} 
    
    Two linear time-invariant systems with transfer functions 
    \begin{align*}
    G_{1}\brak{s} = \frac{10}{s^{2} + s + 1} 
    \end{align*}
    and
    \begin{align*}
    G_{2}\brak{s} = \frac{10}{s^{2}+s\sqrt{10} +10}
    \end{align*}
    have unit step responses $y_{1}\brak{t}$ and $y_{2}\brak{t}$, respectively. Which of the following statements is/are true?
    \begin{enumerate}
    \item $y_{1}\brak{t}$ and $y_{2}\brak{t}$ have the same percentage peak overshoot.\\
    \item $y_{1}\brak{t}$ and $y_{2}\brak{t}$ have the same steady state values.\\
    \item $y_{1}\brak{t}$ and $y_{2}\brak{t}$ have the same damped frequency of oscillation.\\
    \item $y_{1}\brak{t}$ and $y_{2}\brak{t}$ have the same $2\%$ settling time.\\
    \end{enumerate}
    \solution
    \begin{table}[h!]
 	\centering
 	\resizebox{6 cm}{!}{
 		\begin{tabular}{|c|c|c|c|}
\hline
    $\zeta$ & \textbf{Pole Location} & \textbf{Referred to as} & \textbf{Condition} \\
    \hline
    $\zeta > 1$ & Different locations on&&\\& the negative real axis & Overdamped & $R > 2\sqrt{\frac{L}{C}}$\\
    \hline
    $\zeta = 1$ & Coincide on&&\\& the negative real axis & Critically Damped & $R = 2\sqrt{\frac{L}{C}}$\\
    \hline
    $\zeta < 1$ & Complex Conjugate poles in&&\\& the left half of s-plane & Underdamped & $R < 2\sqrt{\frac{L}{C}}$\\
    \hline
\end{tabular}

 	}
 	\caption{Given Parameters}
 	\label{tab:msmECgate50tab1}
     \end{table} 
    The general second-order transfer function is given by:
    \begin{align}
    G\brak{s} = \frac{\omega_n^2}{s^2 + 2\zeta\omega_n s + \omega_n^2}
    \end{align}
    After comparing the coefficients of $G_{1}\brak{s}$ and $G_{2}\brak{s}$,
    \begin{table}[h!]
 	\centering
 	\resizebox{6 cm}{!}{
 		\begin{tabular}{|p{2cm}|p{2.80cm}|p{2.70cm}|}
    \hline
    Symbol&Value&Description\\ \hline
    $$V\brak{t}$$&$$\sin{t}$$&Time varying voltage source\\\hline
    $$I\brak{t}$$&$$\sin{t-\frac{\pi}{4}}$$&Current flowing in the circuit\\\hline
    $$R$$&$$1\ohm$$&Resistor in series to Z\\\hline
    $$Z$$&$$Z$$&Circuit element\\\hline
    \end{tabular}
 	}
 	\caption{Given Parameters}
 	\label{tab:msmECgate50tab2}
     \end{table} 
    as $\zeta = \frac{1}{2}$ is less than 1, the system is underdamped.
    \begin{align}
    Y\brak{s} &= X\brak{s} G\brak{s}\\
    &= \frac{1}{s} \brak{\frac{\omega_n^2}{s^2 + 2\zeta\omega_n s + \omega_n^2}}  
    \end{align}
    Applying inverse laplace transform,
    \begin{equation}
    y(t) = 1 - \frac{e^{-\zeta \omega_n t}}{1 - \zeta^2} \sin(\omega_d t + \phi)
    \label{eq:EC50msm}
    \end{equation}
    where $\omega_{d}$ is the damped frequency of oscillation.
    \begin{equation}
    \omega_{d} = \omega_{n}\sqrt{1 - {\zeta}^2}
    \label{eq:EC50msm2eq}
    \end{equation}
    The percentage peak overshoot $\brak{PO}$:
    \begin{equation}
    PO = \left( \frac{y_{\text{max}} - y_{\text{ss}}}{y_{\text{ss}}} \right) \times 100\%
    \label{eq:EC50msm1eq}
    \end{equation}
    $y_{\text{max}}$ is obtained by differentiating~\eqref{eq:EC50msm} with respect to time and equating it to zero, substituting the value in~\eqref{eq:EC50msm},
    \begin{align}
    y_{\text{max}} = 1 + \frac{1}{\sqrt{1-{\zeta}^2}}
    \end{align}
    $y_{\text{ss}}$ is obtained by final value theorem,
    \begin{align}
    y_{\text{ss}} &= \lim_{{s \to 0}} sY(s)\\
    &= \lim_{{s \to 0}} s\frac{\omega_n^2}{s^2 + 2\zeta\omega_n s + \omega_n^2} \frac{1}{s}\\
    &= 1
    \end{align} 
    Substituting the values of $y_{\text{max}}$ and $y_{\text{ss}}$ in~\eqref{eq:EC50msm1eq}, 
    \begin{align}
    PO = \frac{1}{\sqrt{1-{\zeta}^2}} \times 100\%
    \end{align}
    $y_{1}\brak{t}$ and $y_{2}\brak{t}$ have same $\zeta$, they have same percentage peak overshoot.So, option $\brak{1}$ is correct.\\
    The steady state value of $y\brak{t}$ is given by final value theorem:
    \begin{align}
    y_{1ss} &= \lim_{{s \to 0}} sY_{1}(s)\\
    &= \lim_{{s \to 0}} s \frac{10}{s^{2} + s + 1}  \frac{1}{s}\\
    &= 10\\
    y_{2ss} &= \lim_{{s \to 0}} sY_{2}(s)\\
    &= \lim_{{s \to 0}} s \frac{10}{s^{2}+s\sqrt{10} +10}  \frac{1}{s}\\
    &= 1
    \end{align} 
    as both the unit step responses have different steady state values, option $\brak{2}$ is incorrect.\\
    From~\eqref{eq:EC50msm1eq}, as $\omega_{n}$ is different for $y_{1}\brak{t}$ and $y_{2}\brak{t}$, they have different damped frequency of oscillation. Hence option $\brak{3}$ is incorrect.\\
    Settling time $T_s$:
    \begin{align}
    T_s = \frac{4}{\zeta \omega_n}
    \end{align}
    As, $\omega_{n}$ is different for $y_{1}\brak{t}$ and $y_{2}\brak{t}$, they have different $2\%$ settling time, Hence option $\brak{4}$ is incorrect.\\
    So, only option $\brak{1}$ is correct.   
\end{document}

