\begin{enumerate}[label=\thechapter.\arabic*,ref=\thechapter.\theenumi]

\item Consider the differential equation $\frac{d^2y}{dx^2}-2\frac{dy}{dx}+y=0$. The boundary conditions are $y=0$ and $\frac{dy}{dx}=1$ at $x=0$. Then the value of $y$ at $x=\frac{1}{2}$ \hfill (GATE AE 2022)\\
\solution
\input{2022/AE/37/ae_37.tex}
\pagebreak

\item  A process described by the transfer function
\begin{align}
    G_p(s) = \frac{\brak{10s+1}}{\brak{5s+1}} \nonumber
\end{align}
is forced by a unit step input at time $t = 0$. The output value immediately after the unit step input (at $t = 0^+$) is ? \hfill(Gate 2022 CH 34)\\
\solution
\iffalse
\let\negmedspace\undefined
\let\negthickspace\undefined
\documentclass[journal,12pt,onecolumn]{IEEEtran}
\usepackage{cite}
\usepackage{amsmath,amssymb,amsfonts,amsthm}
\usepackage{algorithmic}
\usepackage{graphicx}
\usepackage{textcomp}
\usepackage{xcolor}
\usepackage{txfonts}
\usepackage{listings}
\usepackage{enumitem}
\usepackage{mathtools}
\usepackage{gensymb}
\usepackage[breaklinks=true]{hyperref}
\usepackage{tkz-euclide} % loads  TikZ and tkz-base
\usepackage{listings}
\usepackage{gvv}
\usepackage{circuitikz}

\newtheorem{theorem}{Theorem}[section]
\newtheorem{problem}{Problem}
\newtheorem{proposition}{Proposition}[section]
\newtheorem{lemma}{Lemma}[section]
\newtheorem{corollary}[theorem]{Corollary}
\newtheorem{example}{Example}[section]
\newtheorem{definition}[problem]{Definition}

\newcommand{\BEQA}{\begin{eqnarray}}
\newcommand{\EEQA}{\end{eqnarray}}
\newcommand{\define}{\stackrel{\triangle}{=}}
\theoremstyle{remark}
\newtheorem{rem}{Remark}

\graphicspath{./figs/}

%\bibliographystyle{ieeetr}
\begin{document}
%

\bibliographystyle{IEEEtran}


\vspace{3cm}

\title{
	%	\logo{
	Gate Assignment

	\large{EE:1205 Signals and Systems}

	Indian Institute of Technology, Hyderabad
	%	}
}
\author{Kunal Thorawade

EE23BTECH11035
}	
\maketitle


%\tableofcontents

\bigskip
 
 \renewcommand{\thefigure}{\theenumi}
 \renewcommand{\thetable}{\arabic{table}}
 \renewcommand{\thefigure}{\arabic{figure}}
 %\renewcommand{\theequation}{\theenumi}

 \textbf{Question}:
 A Spectrometer is used to detect plasma oscillations in a sample. The spectrometer 
 can work in the range of $3 \times 10^{12}$ rad s$^{-1}$ to $30 \times 10^{12}$ rad s$^{-1}$. The minimum carrier concentration that can be detected by using this spectrometer is $n \times 10^{21}$ m$^{-3}$. The value of $n$ is \underline{\hspace{2cm}}. (Round off to two places)
 (Charge on electron $= -1.6 \times 10^{-19} $ C$^{-1}$, mass of electron = $9.1 \times 10^{-31}$ kg and $\epsilon_0 = 8.85 \times 10^{-12}$ C$^{2}$ N$^{-1}$ m$^{-2}$ ) \hfill(GATE PH 35 2022)\\
 \solution 
 \fi
 \begin{table}[ht]
	  \centering
	    \begin{tabular}{|c|c|c|}
		        \hline
			   \textbf{ Parameter} & \textbf{Value} & \textbf{Description} \\
			       \hline
			           $\omega_{p1}$ & $3 \times 10^{12}$ rad s$^{-1}$ & Lower bound of plasma frequency \\
				       \hline
				            $\omega_{p2}$ & $30 \times 10^{12}$ rad s$^{-1}$ & Upper bound of plasma frequency \\
					        \hline
						    $\Delta\omega_p = \omega_{p2} - \omega_{p1}$  & $27 \times 10^{12}$ rad s$^{-1}$ & Plasma Frequency  \\
						        \hline
							    $n_0$ &  $n \times 10^{21}$ &  Minimum carrier concentration \\
							        \hline
								    $e$ &  $-1.6 \times 10^{-19}$ & Charge on electron \\
								        \hline
									    $m$ & $9.1 \times 10^{-31}$ & Mass of electron \\
									        \hline
										  \end{tabular}
										    \vspace{2mm}
										      \caption{Parameter Table}
										        \label{22_PH_35}
\end{table}

 \begin{align}
	     \Delta\omega_p &= \sqrt{\frac{n_0e^2}{m\epsilon_0}} \\
	         \implies n_0 &= \frac{\brak{\Delta\omega_p}^2m\epsilon_0}{e^2} \\
		     n_0 &= \frac{\brak{27 \times 10^{12}}^2 \times \brak{9.1 \times 10^{31} }\times \brak{8.85 \times 10^{-12}}}{\brak{-1.6 \times 10^{-19}}^2} \\
		         \therefore n_0 &= 2.83 \times 10^{21} \text{m}^{-3} \\
			     n &= n_0 \times 10^{-21} \\
			         \therefore n &= 2.83
 \end{align}
 \begin{figure}[ht]
	     \centering
	         \includegraphics[width = \columnwidth]{2022/PH/35/figs/fig1.jpg}
		     \caption{Plot of Spectrometer response vs Plasma frequency}
		         \label{fig2.PH.35}
 \end{figure}

\pagebreak
\item The transfer function of a real system $H(S)$ is given as:
\begin{align}
    H(s) = \frac{As + B}{s^2 + Cs + D}\nonumber
\end{align}
where $A, B, C$ and $D$ are positive constants. This system cannot operate as
\begin{enumerate}[label={(\Alph*)}]
    \item Low pass filter
    \item High pass filter
    \item Band pass filter
    \item An Integrator
\end{enumerate}\hfill(GATE EE 11 2022)

\solution
\iffalse
\let\negmedspace\undefined
\let\negthickspace\undefined
\documentclass[journal,12pt,onecolumn]{IEEEtran}
\usepackage{cite}
\usepackage{amsmath,amssymb,amsfonts,amsthm}
\usepackage{algorithmic}
\usepackage{graphicx}
\usepackage{textcomp}
\usepackage{xcolor}
\usepackage{txfonts}
\usepackage{listings}
\usepackage{enumitem}
\usepackage{mathtools}
\usepackage{gensymb}
\usepackage[breaklinks=true]{hyperref}
\usepackage{tkz-euclide} % loads  TikZ and tkz-base
\usepackage{listings}
\usepackage{gvv}
\usepackage{circuitikz}

\newtheorem{theorem}{Theorem}[section]
\newtheorem{problem}{Problem}
\newtheorem{proposition}{Proposition}[section]
\newtheorem{lemma}{Lemma}[section]
\newtheorem{corollary}[theorem]{Corollary}
\newtheorem{example}{Example}[section]
\newtheorem{definition}[problem]{Definition}

\newcommand{\BEQA}{\begin{eqnarray}}
\newcommand{\EEQA}{\end{eqnarray}}
\newcommand{\define}{\stackrel{\triangle}{=}}
\theoremstyle{remark}
\newtheorem{rem}{Remark}

\graphicspath{./figs/}

%\bibliographystyle{ieeetr}
\begin{document}
%

\bibliographystyle{IEEEtran}


\vspace{3cm}

\title{
	%	\logo{
	Gate Assignment

	\large{EE:1205 Signals and Systems}

	Indian Institute of Technology, Hyderabad
	%	}
}
\author{Kunal Thorawade

EE23BTECH11035
}	
\maketitle


%\tableofcontents

\bigskip
 
 \renewcommand{\thefigure}{\theenumi}
 \renewcommand{\thetable}{\arabic{table}}
 \renewcommand{\thefigure}{\arabic{figure}}
 %\renewcommand{\theequation}{\theenumi}

 \textbf{Question}:
 A Spectrometer is used to detect plasma oscillations in a sample. The spectrometer 
 can work in the range of $3 \times 10^{12}$ rad s$^{-1}$ to $30 \times 10^{12}$ rad s$^{-1}$. The minimum carrier concentration that can be detected by using this spectrometer is $n \times 10^{21}$ m$^{-3}$. The value of $n$ is \underline{\hspace{2cm}}. (Round off to two places)
 (Charge on electron $= -1.6 \times 10^{-19} $ C$^{-1}$, mass of electron = $9.1 \times 10^{-31}$ kg and $\epsilon_0 = 8.85 \times 10^{-12}$ C$^{2}$ N$^{-1}$ m$^{-2}$ ) \hfill(GATE PH 35 2022)\\
 \solution 
 \fi
 \begin{table}[ht]
	  \centering
	    \begin{tabular}{|c|c|c|}
		        \hline
			   \textbf{ Parameter} & \textbf{Value} & \textbf{Description} \\
			       \hline
			           $\omega_{p1}$ & $3 \times 10^{12}$ rad s$^{-1}$ & Lower bound of plasma frequency \\
				       \hline
				            $\omega_{p2}$ & $30 \times 10^{12}$ rad s$^{-1}$ & Upper bound of plasma frequency \\
					        \hline
						    $\Delta\omega_p = \omega_{p2} - \omega_{p1}$  & $27 \times 10^{12}$ rad s$^{-1}$ & Plasma Frequency  \\
						        \hline
							    $n_0$ &  $n \times 10^{21}$ &  Minimum carrier concentration \\
							        \hline
								    $e$ &  $-1.6 \times 10^{-19}$ & Charge on electron \\
								        \hline
									    $m$ & $9.1 \times 10^{-31}$ & Mass of electron \\
									        \hline
										  \end{tabular}
										    \vspace{2mm}
										      \caption{Parameter Table}
										        \label{22_PH_35}
\end{table}

 \begin{align}
	     \Delta\omega_p &= \sqrt{\frac{n_0e^2}{m\epsilon_0}} \\
	         \implies n_0 &= \frac{\brak{\Delta\omega_p}^2m\epsilon_0}{e^2} \\
		     n_0 &= \frac{\brak{27 \times 10^{12}}^2 \times \brak{9.1 \times 10^{31} }\times \brak{8.85 \times 10^{-12}}}{\brak{-1.6 \times 10^{-19}}^2} \\
		         \therefore n_0 &= 2.83 \times 10^{21} \text{m}^{-3} \\
			     n &= n_0 \times 10^{-21} \\
			         \therefore n &= 2.83
 \end{align}
 \begin{figure}[ht]
	     \centering
	         \includegraphics[width = \columnwidth]{2022/PH/35/figs/fig1.jpg}
		     \caption{Plot of Spectrometer response vs Plasma frequency}
		         \label{fig2.PH.35}
 \end{figure}

\pagebreak

\item In a circuit, there is a series connection of an ideal resistor and an ideal capacitor.
The conduction current (in Amperes) through the resistor is $2\sin\brak{t + \frac{\pi}{2}}$. The displacement current (in Amperes) through the capacitor is \rule{1cm}{0.15mm}.\\ 
\begin{enumerate}[label=(\Alph*)]
    \item $2\sin\brak{t}$
    \item $2\sin\brak{t+\pi}$
    \item $2\sin\brak{t +\frac{\pi}{2}}$
    \item $0$
\end{enumerate}
\hfill(GATE 2022 EC 24)\\
\solution
\iffalse
\let\negmedspace\undefined
\let\negthickspace\undefined
\documentclass[journal,12pt,onecolumn]{IEEEtran}
\usepackage{cite}
\usepackage{amsmath,amssymb,amsfonts,amsthm}
\usepackage{algorithmic}
\usepackage{graphicx}
\usepackage{textcomp}
\usepackage{xcolor}
\usepackage{txfonts}
\usepackage{listings}
\usepackage{enumitem}
\usepackage{mathtools}
\usepackage{gensymb}
\usepackage[breaklinks=true]{hyperref}
\usepackage{tkz-euclide} % loads  TikZ and tkz-base
\usepackage{listings}
\usepackage{gvv}
\usepackage{circuitikz}

\newtheorem{theorem}{Theorem}[section]
\newtheorem{problem}{Problem}
\newtheorem{proposition}{Proposition}[section]
\newtheorem{lemma}{Lemma}[section]
\newtheorem{corollary}[theorem]{Corollary}
\newtheorem{example}{Example}[section]
\newtheorem{definition}[problem]{Definition}

\newcommand{\BEQA}{\begin{eqnarray}}
\newcommand{\EEQA}{\end{eqnarray}}
\newcommand{\define}{\stackrel{\triangle}{=}}
\theoremstyle{remark}
\newtheorem{rem}{Remark}

\graphicspath{./figs/}

%\bibliographystyle{ieeetr}
\begin{document}
%

\bibliographystyle{IEEEtran}


\vspace{3cm}

\title{
	%	\logo{
	Gate Assignment

	\large{EE:1205 Signals and Systems}

	Indian Institute of Technology, Hyderabad
	%	}
}
\author{Kunal Thorawade

EE23BTECH11035
}	
\maketitle


%\tableofcontents

\bigskip
 
 \renewcommand{\thefigure}{\theenumi}
 \renewcommand{\thetable}{\arabic{table}}
 \renewcommand{\thefigure}{\arabic{figure}}
 %\renewcommand{\theequation}{\theenumi}

 \textbf{Question}:
 A Spectrometer is used to detect plasma oscillations in a sample. The spectrometer 
 can work in the range of $3 \times 10^{12}$ rad s$^{-1}$ to $30 \times 10^{12}$ rad s$^{-1}$. The minimum carrier concentration that can be detected by using this spectrometer is $n \times 10^{21}$ m$^{-3}$. The value of $n$ is \underline{\hspace{2cm}}. (Round off to two places)
 (Charge on electron $= -1.6 \times 10^{-19} $ C$^{-1}$, mass of electron = $9.1 \times 10^{-31}$ kg and $\epsilon_0 = 8.85 \times 10^{-12}$ C$^{2}$ N$^{-1}$ m$^{-2}$ ) \hfill(GATE PH 35 2022)\\
 \solution 
 \fi
 \begin{table}[ht]
	  \centering
	    \begin{tabular}{|c|c|c|}
		        \hline
			   \textbf{ Parameter} & \textbf{Value} & \textbf{Description} \\
			       \hline
			           $\omega_{p1}$ & $3 \times 10^{12}$ rad s$^{-1}$ & Lower bound of plasma frequency \\
				       \hline
				            $\omega_{p2}$ & $30 \times 10^{12}$ rad s$^{-1}$ & Upper bound of plasma frequency \\
					        \hline
						    $\Delta\omega_p = \omega_{p2} - \omega_{p1}$  & $27 \times 10^{12}$ rad s$^{-1}$ & Plasma Frequency  \\
						        \hline
							    $n_0$ &  $n \times 10^{21}$ &  Minimum carrier concentration \\
							        \hline
								    $e$ &  $-1.6 \times 10^{-19}$ & Charge on electron \\
								        \hline
									    $m$ & $9.1 \times 10^{-31}$ & Mass of electron \\
									        \hline
										  \end{tabular}
										    \vspace{2mm}
										      \caption{Parameter Table}
										        \label{22_PH_35}
\end{table}

 \begin{align}
	     \Delta\omega_p &= \sqrt{\frac{n_0e^2}{m\epsilon_0}} \\
	         \implies n_0 &= \frac{\brak{\Delta\omega_p}^2m\epsilon_0}{e^2} \\
		     n_0 &= \frac{\brak{27 \times 10^{12}}^2 \times \brak{9.1 \times 10^{31} }\times \brak{8.85 \times 10^{-12}}}{\brak{-1.6 \times 10^{-19}}^2} \\
		         \therefore n_0 &= 2.83 \times 10^{21} \text{m}^{-3} \\
			     n &= n_0 \times 10^{-21} \\
			         \therefore n &= 2.83
 \end{align}
 \begin{figure}[ht]
	     \centering
	         \includegraphics[width = \columnwidth]{2022/PH/35/figs/fig1.jpg}
		     \caption{Plot of Spectrometer response vs Plasma frequency}
		         \label{fig2.PH.35}
 \end{figure}

\newpage

\item Given, $y=f\brak{x}$; $\frac{d^2y}{dx2}+4y=0; y\brak{0}=0; \frac{dy}{dx}\brak{0}=1$. The problem is a/an \\
\begin{enumerate}[label=(\alph*)]
    \item initial value problem having soluition $y=x$
    \item boundary value problem having soluition $y=x$
    \item initial value problem having soluition $y=\frac{1}{2}\sin 2x$
    \item boundary value problem having soluition {$y=\frac{1}{2}\sin 2x$}
\end{enumerate} \hfill(GATE 2022 ES)    \\
\solution
\input{2022/ES/36/gate4.tex}
\newpage
\item Consider the circuit shown in the figure with input V(t) in volts.The sinusoidal steady state current I(t) flowing through the circuit is shown graphically(where t is in seconds). The circuit element Z can be\rule{1.5cm}{0.15mm}.\hfill{GATE-2022-EC-39}
\begin{enumerate}
    \item a capacitor of 1 F
    \item an inductor of 1 H
    \item a capacitor of $\sqrt{3}$ H
    \item an inductor of $\sqrt{3}$ H
\end{enumerate}
\begin{figure}[ht]
    \centering
    \includegraphics[width=\columnwidth]{2022/39/figs/Figure_1.png}
    \label{fig:GATE.2022.EC.39.1}
\end{figure}
\solution
\iffalse
\let\negmedspace\undefined
\let\negthickspace\undefined
\documentclass[journal,12pt,onecolumn]{IEEEtran}
\usepackage{cite}
\usepackage{amsmath,amssymb,amsfonts,amsthm}
\usepackage{algorithmic}
\usepackage{graphicx}
\usepackage{textcomp}
\usepackage{xcolor}
\usepackage{txfonts}
\usepackage{listings}
\usepackage{enumitem}
\usepackage{mathtools}
\usepackage{gensymb}
\usepackage[breaklinks=true]{hyperref}
\usepackage{tkz-euclide} % loads  TikZ and tkz-base
\usepackage{listings}
\usepackage{gvv}
\usepackage{circuitikz}

\newtheorem{theorem}{Theorem}[section]
\newtheorem{problem}{Problem}
\newtheorem{proposition}{Proposition}[section]
\newtheorem{lemma}{Lemma}[section]
\newtheorem{corollary}[theorem]{Corollary}
\newtheorem{example}{Example}[section]
\newtheorem{definition}[problem]{Definition}

\newcommand{\BEQA}{\begin{eqnarray}}
\newcommand{\EEQA}{\end{eqnarray}}
\newcommand{\define}{\stackrel{\triangle}{=}}
\theoremstyle{remark}
\newtheorem{rem}{Remark}

\graphicspath{./figs/}

%\bibliographystyle{ieeetr}
\begin{document}
%

\bibliographystyle{IEEEtran}


\vspace{3cm}

\title{
	%	\logo{
	Gate Assignment

	\large{EE:1205 Signals and Systems}

	Indian Institute of Technology, Hyderabad
	%	}
}
\author{Kunal Thorawade

EE23BTECH11035
}	
\maketitle


%\tableofcontents

\bigskip
 
 \renewcommand{\thefigure}{\theenumi}
 \renewcommand{\thetable}{\arabic{table}}
 \renewcommand{\thefigure}{\arabic{figure}}
 %\renewcommand{\theequation}{\theenumi}

 \textbf{Question}:
 A Spectrometer is used to detect plasma oscillations in a sample. The spectrometer 
 can work in the range of $3 \times 10^{12}$ rad s$^{-1}$ to $30 \times 10^{12}$ rad s$^{-1}$. The minimum carrier concentration that can be detected by using this spectrometer is $n \times 10^{21}$ m$^{-3}$. The value of $n$ is \underline{\hspace{2cm}}. (Round off to two places)
 (Charge on electron $= -1.6 \times 10^{-19} $ C$^{-1}$, mass of electron = $9.1 \times 10^{-31}$ kg and $\epsilon_0 = 8.85 \times 10^{-12}$ C$^{2}$ N$^{-1}$ m$^{-2}$ ) \hfill(GATE PH 35 2022)\\
 \solution 
 \fi
 \begin{table}[ht]
	  \centering
	    \begin{tabular}{|c|c|c|}
		        \hline
			   \textbf{ Parameter} & \textbf{Value} & \textbf{Description} \\
			       \hline
			           $\omega_{p1}$ & $3 \times 10^{12}$ rad s$^{-1}$ & Lower bound of plasma frequency \\
				       \hline
				            $\omega_{p2}$ & $30 \times 10^{12}$ rad s$^{-1}$ & Upper bound of plasma frequency \\
					        \hline
						    $\Delta\omega_p = \omega_{p2} - \omega_{p1}$  & $27 \times 10^{12}$ rad s$^{-1}$ & Plasma Frequency  \\
						        \hline
							    $n_0$ &  $n \times 10^{21}$ &  Minimum carrier concentration \\
							        \hline
								    $e$ &  $-1.6 \times 10^{-19}$ & Charge on electron \\
								        \hline
									    $m$ & $9.1 \times 10^{-31}$ & Mass of electron \\
									        \hline
										  \end{tabular}
										    \vspace{2mm}
										      \caption{Parameter Table}
										        \label{22_PH_35}
\end{table}

 \begin{align}
	     \Delta\omega_p &= \sqrt{\frac{n_0e^2}{m\epsilon_0}} \\
	         \implies n_0 &= \frac{\brak{\Delta\omega_p}^2m\epsilon_0}{e^2} \\
		     n_0 &= \frac{\brak{27 \times 10^{12}}^2 \times \brak{9.1 \times 10^{31} }\times \brak{8.85 \times 10^{-12}}}{\brak{-1.6 \times 10^{-19}}^2} \\
		         \therefore n_0 &= 2.83 \times 10^{21} \text{m}^{-3} \\
			     n &= n_0 \times 10^{-21} \\
			         \therefore n &= 2.83
 \end{align}
 \begin{figure}[ht]
	     \centering
	         \includegraphics[width = \columnwidth]{2022/PH/35/figs/fig1.jpg}
		     \caption{Plot of Spectrometer response vs Plasma frequency}
		         \label{fig2.PH.35}
 \end{figure}

\pagebreak
\newpage
\end{enumerate}
